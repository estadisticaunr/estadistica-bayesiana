\usepackage{fancyhdr}                  % Para usar 'fancypagestyle'

\renewcommand{\headrulewidth}{1pt}     % Linea horizontal en el header
\renewcommand{\footrulewidth}{0.4pt}   % Linea horizontal en el footer

% Cambia el estilo de la página y permite escribir texto en el header y footer
% Hay un estilo para la primera página, y un estilo apra el resto
\fancypagestyle{plain} {
    \lhead{
        \begin{minipage}{6cm}
            \includegraphics[width=5.5cm]{$logo$} 
        \end{minipage}
    }
    \chead{}
    \rhead{
        \parbox{10cm}{
            \textsc{Facultad de Ciencias Económicas y Estadística\\Licenciatura en Estadística\\Estadística Bayesiana}
        }
    }
    \lfoot{$practica$}
    \rfoot{Página \thepage}
}

% Para el resto de las páginas, el estilo es el siguiente
\pagestyle{fancy}
\lhead{$course$}
\rhead{\number$year$}
\lfoot{$practica$}
\rfoot{Página \thepage}
\cfoot{}

% Cambiar como funciona el maketitle. Por ejemplo, omite la fecha.
\def\maketitle{%
    \thispagestyle{plain}
    \hspace{100pt}% No sé por que se necesita espacio horizontal
    \begin{center}
        \vspace{0.5cm}
		\textbf{\LARGE $practica$}
        \vspace{0.5cm}
	\end{center}
}